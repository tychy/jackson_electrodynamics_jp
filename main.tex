\documentclass{article}
\begin{document}
\section  {question1.1}
(a) この問いでは、系は静的な平衡にあるという仮定がされている。
静的な平衡では自由電子がその場所に電場がない場合にのみその場所に存在することができる。
そうでなければ、自由電子は電場によって加速されるだろう。故に、導体内部の電場は0である。
導体の表面では、電荷は導体の表面と垂直な方向へ束縛されている。
そして導体の内部、外部で存在しうる電場は、導体表面に対して垂直な電場のみである。
\\
表面がガウス平面内にある任意の閉曲面においてガウスの法則を用いて、
\[
  \oint_S \mathbf{E} \cdot \mathbf{n} da = \frac{q}{\epsilon_0}
\]
ここで閉曲面の内部ではEは0なので、その表面積分も当然0である。よって、導体内部ではq=0であることがわかった。
\\
(b) 

\end{document}
